%%% ====================================================================
\documentclass[fleqn]{article}

\usepackage[utf8x]{vietnam}

\title{Kiểm tra môi trường `subequations'}
\author{kyanh $\langle$\texttt{mailto:kyanh@o2.pl}$\rangle$}
\date{\today}

\usepackage{amsmath}
\numberwithin{equation}{section}

\newcommand{\env}[1]{{\normalfont\texttt{#1}}}

\AtEndDocument{\multipasswarning}
\newcommand{\multipasswarning}{%
  \clearpage
  \typeout{%
  **********************************************************************}
  \typeout{%
  Note: This document needs to run through LaTeX three times, instead of}
  \typeout{%
  the usual two, to resolve indirect cross-references.}
  \typeout{%
  **********************************************************************}
}

\makeatletter
%    Omit the warning message if three passes have been completed (on
%    first pass \ref{check} is undefined and it is set to 0; on second
%    pass \ref{check} is 0 and set to 1; on third pass it is 1).
\def\checkref{\begingroup
  \@ifundefined{r@check}{\def\@currentlabel{0}}{\def\@currentlabel{1}}%
  \ifnum1=0\expandafter\@firstoftwo\r@check\relax
    \global\let\multipasswarning\relax
  \fi
  \label{check}%
  \endgroup}
\makeatother

\parindent0pt

\begin{document}

\maketitle

\section{Đánh số thứ cấp}

Đây là phương trình đầu tiên.
\begin{equation}\label{e:previous}
A=B
\end{equation}
Đó là phương trình \eqref{e:previous}.

Ta sẽ sử dụng cùng chỉ số, nhưng với dấu phảy (đánh số thứ cấp).\checkref
\begin{equation}
\tag{\ref{e:previous}$'$}\label{e:prevprime}
C=D
\end{equation}
Phương trình trên là \eqref{e:prevprime}.

%% Notice, by the way, that when a \verb"\ref" occurs inside a \verb"\tag",
%% and that \verb"\tag" is then \verb"\label"'d, a \verb"\ref" for the the
%% second \verb"\label" requires \emph{three} runs of \LaTeX{} in order to
%% get the proper value. (If you run through the logic of \LaTeX{}'s
%% cross-referencing mechanisms as they apply in this case, you will see
%% that this is necessary.)
\medskip 
Chú ý rằng, khi sử dụng \verb"\ref" bên trong lệnh \verb"\tag",
và sau đó \verb"\tag" sinh ra được dùng làm nhãn với \verb"\label", bạn
cần phải biên dịch tài liệu ít nhất ba lần để có các tham khảo chính xác.
Điều này là tự nhiên, nếu bạn hiểu nguyên lý thực hiện tham khảo chéo của \LaTeX{}.


\section{Đánh số phương trình con}

%Here is a,b,c sub-numbering.
Đây là ví dụ.
\begin{subequations}
\begin{eqnarray}
A&=&B\\
D&=&C \label{e:middle}\\
E&=&F
\end{eqnarray}
\end{subequations}
%% That was produced with the \env{eqnarray} environment; the middle line
%% was labeled as \eqref{e:middle}.
Ba phương trình trên được bố trí nhờ môi trường \env{eqnarray};
môi trường giữa được đánh chỉ số \eqref{e:middle}.

%% An equation following the end of the \verb"subequations" environment
%% should revert to normal numbering:
\medskip
Phương trình theo ngay sau môi trường \env{subequations} sẽ được
đánh số theo cách tự nhiên:
\begin{equation}\label{e:check}
H<K
\end{equation}
%A check on the labeling: that was equation~\eqref{e:check}.
Thử kiểm tra lại việc đánh nhãn: phương trình trên là \eqref{e:check}.

%% The sub-numbered equations can be spread out through the text, like
%% this:
\medskip
Bên trong môi trường \env{subequations} có thể xen kẽ các môi trường
phương trình và \verb"\text", nội dung khác. Ví dụ
\begin{subequations}
\begin{equation}
A=B
\end{equation}
%% The \verb"subequations" environment can span arbitrary text between
%% subsidiary equations. The only restriction is that if there are any
%% numbered equations inside the \verb"subequations" environment that break
%% out of the subequation numbering sequence, they would have to be handled
%% specially.
Hạn chế duy nhất là các môi trường bên trong \env{subequations} được đánh
số tuần tự, do đó, nếu muốn có một chỉ số khác, bạn phải xử lý riêng.
\begin{equation}
D=C \label{e:newmiddle}
\end{equation}
Đây là phần nội dung khác bên trong môi trường \env{subequations}.
\begin{equation}
E=F
\end{equation}
\end{subequations}
%Label check: the middle one was \eqref{e:newmiddle}
Kiểm tra việc đánh nhãn: phương trình giữa là \eqref{e:newmiddle}.

%A final equation for a numbering check.
Phương trình theo sau môi trường \env{equations} ở trên:
\begin{equation}\label{e:final}
G=H
\end{equation}
%That equation was labeled as \eqref{e:final}.
Phương trình này được đánh nhãn \eqref{e:final}.

%% \section{Tests of \env{align}, \env{gather}, and other
%% AMS-\protect\LaTeX{} environments}
\subsection{Kiểm tra với các môi trường khác}


%% The \env{align} environment:
Sau đây là môi trường \env{align}:
\begin{subequations}
\begin{align}
\label{align:a}A+B&=B+A\\
\label{align:b}C&=D+E\\
\label{align:c}E&=F
\end{align}
\end{subequations}
%Label check: that was \eqref{align:a}, \eqref{align:b}, and
Các nhãn đã được dùng lần lượt là \eqref{align:a}, \eqref{align:b} và
\eqref{align:c}.

\medskip
%The \env{align} environment again:
Cùng với môi trường \env{align} lần nữa:
\begin{subequations}
\begin{align}
\label{xalign:a}A+B&=B&     B&=B+A\\
\label{xalign:b}C&=D+E&     C\oplus D&=E\\
\label{xalign:c}E&=F&       E'&=F'
\end{align}
\end{subequations}
%Label check: that was 
Các nhãn vừa được dùng là
\eqref{xalign:a}, \eqref{xalign:b}, %and
\eqref{xalign:c}.

\medskip
%% The \env{gather} environment. For the third line we refer to one of the
%% numbers in the first \env{align} structure.
Bây giờ ta kiểm tra với môi trường \env{gather}.
Ở phương trình thứ ba, ta tham khảo đến một trong các phương trình
đã kiểm tra ở trên với môi trường \env{align}.
\begin{subequations}
\begin{gather}
\label{gather:a}A+B=B\\
\label{gather:b}C=D+E\\
\tag{\ref{align:a}$'$}\label{gather:c}E=F
\end{gather}
\end{subequations}
%Label check: that was
Các nhãn vừa được dùng:
 \eqref{gather:a},
\eqref{gather:b}, và
\eqref{gather:c}.

%% The next \env{subequations} environment encompasses two separate
%% equations. A \env{split} environment:
\medskip
Tiếp theo, môi trường \env{subequations} được kiểm tra với
hai cách bố trí phương trình khác nhau. Đầu tiên là môi trường
\env{split}
\begin{subequations}
\begin{equation}
\label{split:x}
\begin{split}
A&=B+C+F\\
&=G
\end{split}
\end{equation}
%and a \env{multline} environment:
và bây giờ là môi trường \env{multline}:
\begin{multline}\label{multline:x}
A[B]C[D]E[F]G[[H[I]J[K]L[M]N]]=\\
H[I]J[K]L[M]N[O]P[Q]R[S]T[U]V[W]X[Y]Z
\end{multline}
\end{subequations}
%Label check: That was \eqref{split:x} and \eqref{multline:x}.
Các nhãn được dùng là \eqref{split:x} và \eqref{multline:x}
\end{document}
